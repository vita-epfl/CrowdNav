%%%%%%%%%%%%%%%%%%%%%%%%%%%%%%%%%%%%%%%%%%%%%%%%%%%%%%%%%%%%%%%%%%%%%%%%%%%%%%%%
%2345678901234567890123456789012345678901234567890123456789012345678901234567890
%        1         2         3         4         5         6         7         8

\documentclass[letterpaper, 10 pt, conference]{ieeeconf}  % Comment this line out if you need a4paper
%\documentclass[a4paper, 10pt, conference]{ieeeconf}      % Use this line for a4 paper

\IEEEoverridecommandlockouts                              % This command is only needed if 
                                                          % you want to use the \thanks command

\overrideIEEEmargins                                      % Needed to meet printer requirements.

% See the \addtolength command later in the file to balance the column lengths
% on the last page of the document

% The following packages can be found on http:\\www.ctan.org
\usepackage{graphics} % for pdf, bitmapped graphics files
\usepackage{epsfig} % for postscript graphics files
\usepackage{mathptmx} % assumes new font selection scheme installed
\usepackage{times} % assumes new font selection scheme installed
\usepackage{amsmath} % assumes amsmath package installed
\usepackage{amssymb}  % assumes amsmath package installed
% \usepackage{comment}
% \usepackage{algorithm,algpseudocode}
% \usepackage{resizegather}
% \usepackage{flexisym}
\usepackage{dblfloatfix}    % To enable figures at the bottom of pag
\usepackage{subcaption}

% TODO: french letter in name
% \usepackage{inputenc} 

\title{\Large \bf Social RL: Socially Compliant Navigation in Crowds
with Reinforcement Learning}

\author{% <-this % stops a space
\thanks{This work was supported by [?]}% <-this % stops a space
% \thanks{$^{1}$L is with, 
% {\tt\small b.d.researcher@ieee.org}}%
% \thanks{$^{*}$Equal contribution}       
\thanks{Visual Intelligence for Transportation Laboratory, Ecole Polytechnique Federale de Lausanne (EPFL)
, CH-1015 Lausanne,
        {\tt\small \{ \}@epfl.ch}}%
}

\begin{document}

\bstctlcite{IEEEexample:BSTcontrol}

\maketitle
\thispagestyle{empty}
\pagestyle{empty}

\pdfminorversion=4  
%%%%%%%%%%%%%%%%%%%%%%%%%%%%%%%%%%%%%%%%%%%%%%%%%%%%%%%%%%%%%%%%%%%%%%%%%%%%%%%%
\begin{abstract}
This electronic document is a template. The various components of your paper [title, text, heads, etc.] are already defined on the style sheet, as illustrated by the portions given in this document.

\vspace{5cm}

\end{abstract}


%%%%%%%%%%%%%%%%%%%%%%%%%%%%%%%%%%%%%%%%%%%%%%%%%%%%%%%%%%%%%%%%%%%%%%%%%%%%%%%%
\section{INTRODUCTION} \label{sec:intro}

With the rapid growth of machine intelligence, robots are envisioned to expand habitats from isolated environments to social spaces that are shared with humans and other embodied agents. Traditional approaches for robot navigation often view the other moving agents as a static obstacle \cite{fox_dynamic_1997} or react to them through a one-step lookahead \cite{berg_reciprocal_2008}, resulting in short-sighted and unnatural behaviors. In order to navigate through a crowd in a socially compliant manner, the robot needs the ability to understand others' behaviors and take that into account while making its movement decision. 

Navigation with social etiquettes is, however, a very challenging task. As communications among agents (e.g. robots, pedestrians, bikers) are not widely available, the robot needs to perceive and predict others' behaviors that may involve complex interactions. Research works in trajectory prediction have proposed several hand-crafted or data-driven methods to model the agent-agent interactions \cite{helbing_social_1995,alahi_social_2016,vemula_social_2017,gupta_social_2018}. 
Given a predictive model, how to integrate the interaction knowledge into the decision-making process also poses a challenge. 

Earlier works split the prediction and planning into two separate steps, attempting to identify a safe trajectory far distant from the forecasted regions of the other agents \cite{bennewitz_learning_2005,aoude_probabilistically_2013}. However, the untraversable regions covering dynamic obstacles for a few steps can be very large when crowded and eventually cause the freezing robot problem \cite{trautman_unfreezing_2010}. To cope with it, jointly obstacle avoidance methods that plan the paths of the interactive decision-makers simultaneously are proposed, in hope to make rooms for each other cooperatively \cite{trautman_unfreezing_2010}. Though conceptually desirable, these methods suffer from the computational costs as well as the stochasticity of neighbors in practice. 

As an alternative, the reinforcement learning method has been used to train a policy that implicitly encodes the interactions and cooperations among agents. While great progress has been made in recent works \cite{chen_decentralized_2016,chen_socially_2017,long_towards_2017,everett_motion_2018}, their policy models are limited in two aspects. i) The impact of a group of other agents is treated by a simple model, for example, a maximum or LSTM operation over the individual impact, which limits the capacity of distinguishing the different importance for each individual neighbor and representing the joint group impact. ii) They omit the interactions within the neighboring agents when producing the joint neighboring state. These dimension reduction may loss critical interaction features and bounds the cooperative planning-based reasoning in complex and crowded scenes. 

To address these challenges, we propose a novel pooling module for the reinforcement learning method, aiming to learn a socially compliant navigation policy in the complex scenes. Draw aspiration from \cite{gupta_social_2018,vemula_social_2017}, we extract the pair-wise interaction feature between the navigator and the other nearby and employ the soft attention to represent the impact importance. The impact of the adjacent neighbors on a specific agent is modeled by a local occupancy map. Based on a social state reflecting these interactions, the robot is trained to learn the value network which is used to evaluate the scenarios confronting a group of the others controlled by some stochastic policies and navigate through them in a cooperative way. An extensive set of experiments shows that our model can effectively learn a socially-compliant navigation policy in the crowd, outperforming the state-of-the-art methods. 

\section{BACKGROUND} \label{sec:background} 

\subsection{Related Work}

RL, 
\vspace{2cm}

Social interaction, 
\vspace{2cm}

\subsection{Problem Formulation}

\section{APPROACH} \label{sec:approach} 

Formulation, 

\section{RESULTS} \label{sec:results} 

Simulation experiments, 

Real-world experiments, and vidoe. 

\section{CONCLUSION} \label{sec:conclusion} 

In this paper, 

% \subsection{System Structure}

\addtolength{\textheight}{-10cm}   % This command serves to balance the column lengths
                                  % on the last page of the document manually. It shortens
                                  % the textheight of the last page by a suitable amount.
                                  % This command does not take effect until the next page
                                  % so it should come on the page before the last. Make
                                  % sure that you do not shorten the textheight too much.

\section*{ACKNOWLEDGMENT}

We would like to thank ... 

\bibliographystyle{IEEEtran}
\bibliography{vita-social-nav}

% \begin{thebibliography}{99}

% \end{thebibliography}

\end{document}


% @IEEEtranBSTCTL{IEEEexample:BSTcontrol,
%   CTLuse_url = "no",
%   CTLuse_article_number = "yes",
%   CTLuse_paper = "yes",
%   CTLuse_forced_etal = "yes",
%   CTLmax_names_forced_etal = "4",
%   CTLnames_show_etal = "1",
%   CTLuse_alt_spacing = "yes",
%   CTLalt_stretch_factor = "4",
%   CTLdash_repeated_names = "yes",
%   CTLname_latex_cmd = ""
% }