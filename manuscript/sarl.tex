%%%%%%%%%%%%%%%%%%%%%%%%%%%%%%%%%%%%%%%%%%%%%%%%%%%%%%%%%%%%%%%%%%%%%%%%%%%%%%%%
%2345678901234567890123456789012345678901234567890123456789012345678901234567890
%        1         2         3         4         5         6         7         8

\documentclass[letterpaper, 10 pt, conference]{ieeeconf}  % Comment this line out if you need a4paper
%\documentclass[a4paper, 10pt, conference]{ieeeconf}      % Use this line for a4 paper

\IEEEoverridecommandlockouts                              % This command is only needed if 
                                                          % you want to use the \thanks command

\overrideIEEEmargins                                      % Needed to meet printer requirements.

% See the \addtolength command later in the file to balance the column lengths
% on the last page of the document

% The following packages can be found on http:\\www.ctan.org
\usepackage{graphics} % for pdf, bitmapped graphics files
\usepackage{epsfig} % for postscript graphics files
\usepackage{mathptmx} % assumes new font selection scheme installed
\usepackage{times} % assumes new font selection scheme installed
\usepackage{amsmath} % assumes amsmath package installed
\usepackage{amssymb}  % assumes amsmath package installed
% \usepackage{comment}
% \usepackage{algorithm,algpseudocode}
% \usepackage{resizegather}
% \usepackage{flexisym}
\usepackage{dblfloatfix}    % To enable figures at the bottom of pag
\usepackage{subcaption}

% TODO: french letter in name
% \usepackage{inputenc} 

\title{\Large \bf Social RL: Socially Compliant Navigation in Crowds
with Reinforcement Learning}

\author{% <-this % stops a space
\thanks{This work was supported by [?]}% <-this % stops a space
% \thanks{$^{1}$L is with, 
% {\tt\small b.d.researcher@ieee.org}}%
% \thanks{$^{*}$Equal contribution}       
\thanks{Visual Intelligence for Transportation Laboratory, Ecole Polytechnique Federale de Lausanne (EPFL)
, CH-1015 Lausanne,
        {\tt\small \{ \}@epfl.ch}}%
}

\begin{document}

\bstctlcite{IEEEexample:BSTcontrol}

\maketitle
\thispagestyle{empty}
\pagestyle{empty}

\pdfminorversion=4  
%%%%%%%%%%%%%%%%%%%%%%%%%%%%%%%%%%%%%%%%%%%%%%%%%%%%%%%%%%%%%%%%%%%%%%%%%%%%%%%%
\begin{abstract}
This electronic document is a template. The various components of your paper [title, text, heads, etc.] are already defined on the style sheet, as illustrated by the portions given in this document.

\vspace{5cm}

\end{abstract}


%%%%%%%%%%%%%%%%%%%%%%%%%%%%%%%%%%%%%%%%%%%%%%%%%%%%%%%%%%%%%%%%%%%%%%%%%%%%%%%%
\section{INTRODUCTION} \label{sec:intro}

With the rapid growth of machine intelligence, robots are envisioned to expand habitats from isolated environments to social spaces that are shared with humans and other embodied agents. Traditional approaches for robot navigation often consider this as a static or one-step reactive problem [TODO], resulting in short-sighted behaviors. In order to navigate through a crowd in a socially compliant manner, robot needs the ability to understand the others' behavior and take that into account while making its motion decision. 

Autonomous navigation with the socially cooperative thinking is, however, a very challenging task. First, a model is required to perceive and predict others' behaviors, when communication is not widely available amoung agents. Pioneering works have proposed several methods to model the agent-agent interactions \cite{helbing_social_1995}. Though effective in simple scenarios, these hand-crafted feature functions can hardly generalize well to complex scene. Recent works try to learn the social interaction by a data-drive approach and present promising results \cite{alahi_social_2016,vemula_social_2017,gupta_social_2018}. 

Given a predictive model, how to integrate the interaction knowledge into the decision making process poses another challenge. 

A set of methods address this issue by splitting the prediction and planning [TODO]. 

Recent research in 
- Learn a planning model that incorporates social interaction \cite{chen_decentralized_2016,chen_socially_2017,everett_motion_2018}

However, the feature-loss dimension reduction limits its capablity of planning-based reasoning in complex and crowded scenarios. 

\vspace{2cm}

In this work, we propose a model that can effectively address challenge of the social interaction reinforcement leanring. Draw aspiration from, we extend the social pooling module to spatial interaction, and embed it into the rl framework. The propsed model use 

Models that effectively encodes the social interactions for rl. 

\vspace{2cm}

* Voila: here's our awesome solution (additionally) and here's how it works.

Pair-wise interaction feature, soft attention, social pooling, local/global occupancy map.

Capable of accurately estimating the value of the crowd scenarios with respect to social navigation, and effectively learning a socially-compliant navigation policy. 

An extensive set of simulation experiments shows that our model outperforms state-of-the-art methods 


\vspace{10cm}

\section{BACKGROUND} \label{sec:background} 

\subsection{Related Work}

RL, 
\vspace{2cm}

Social interaction, 
\vspace{2cm}

\subsection{Problem Formulation}

\section{APPROACH} \label{sec:approach} 

Formulation, 

\section{RESULTS} \label{sec:results} 

Simulation experiments, 

Real-world experiments, and vidoe. 

\section{CONCLUSION} \label{sec:conclusion} 

In this paper, 

% \subsection{System Structure}

\addtolength{\textheight}{-10cm}   % This command serves to balance the column lengths
                                  % on the last page of the document manually. It shortens
                                  % the textheight of the last page by a suitable amount.
                                  % This command does not take effect until the next page
                                  % so it should come on the page before the last. Make
                                  % sure that you do not shorten the textheight too much.

\section*{ACKNOWLEDGMENT}

We would like to thank ... 

\bibliographystyle{IEEEtran}
\bibliography{vita-social-nav}

% \begin{thebibliography}{99}

% \end{thebibliography}

\end{document}


% @IEEEtranBSTCTL{IEEEexample:BSTcontrol,
%   CTLuse_url = "no",
%   CTLuse_article_number = "yes",
%   CTLuse_paper = "yes",
%   CTLuse_forced_etal = "yes",
%   CTLmax_names_forced_etal = "4",
%   CTLnames_show_etal = "1",
%   CTLuse_alt_spacing = "yes",
%   CTLalt_stretch_factor = "4",
%   CTLdash_repeated_names = "yes",
%   CTLname_latex_cmd = ""
% }